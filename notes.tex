% xelatex

\documentclass{report}

\usepackage{geometry}
\geometry{a4paper, top=1in, bottom=1.5in}
\setlength{\parindent}{0pt}
\setlength{\parskip}{6pt}

\usepackage{amsmath}
\usepackage{amssymb}

\DeclareMathOperator{\Tor}{Tor}
\DeclareMathOperator{\Hom}{Hom}
\DeclareMathOperator{\End}{End}
\DeclareMathOperator{\im}{im}

\begin{document}

\title{Dummit and Foote\\Answers to Exercises}
\author{Marc-André Brochu}
\date{Winter 2019}
\maketitle

\setcounter{chapter}{8}

\chapter{Polynomial Rings}

\section{Polynomial Rings over Fields II}

For the remaining exercises let $F$ be a field, let $F^n$ be the set of all $n$-tuples of elements of $F$ (called \textit{affine $n$-space over F}) and let $R$ be the polynomial ring $F[x_1,x_2,\dots,x_n]$. The elements of $R$ form a ring of $F$-valued functions on $F^n$, where the value of the polynomial $p(x_1,\dots,x_n)$ on the $n$-tuple $(a_1,\dots,a_n)$ is obtained by substituting $a_i$ for $x_i$ for all $i$.

\begin{enumerate}
\item[12.]
\begin{enumerate}
\item
Let $X$ be any given subset of $F^n$. We always have $0_R\in I(X)$ and thus $I(X)$ is never empty. Take any $f,g\in I(X)$. Then for all $a\in X$, $(f+g)(x) = f(x)+g(x)=0$. Thus $I(X)$ is closed under addition. Take some $h\in R$. For all $a\in X$, $(h\cdot f)(x)=h(x)f(x)=0$ which means that $I(X)$ absorbs left multiplication. Because $R$ is commutative, we get that $I(X)$ is an ideal in this ring.

Let $J\subseteq R$ be arbitrarily given. If $a\in V(\langle J\rangle)$, then for all $f\in J\subseteq \langle J\rangle$, we have that $f(a)=0$. Thus $V(\langle J\rangle)\subseteq V(J)$. Now let $a\in V(J)$. Take any $f\in \langle J\rangle$. Then $f$ is a finite combination of $R$-multiples of elements of $J$, i.e. $f=f_1j_1+\dots+f_nj_n$ with $f_i\in R$, $j_i\in J$ and $n\in \mathbb{N}$. So $f(a) = f_1(a)j_1(a)+\dots+f_n(a)j_n(a)$. Since for all $j\in J$, $j(a)=0$, we get that $f(a)=0$ and thus $a\in V(\langle J\rangle)$. Therefore $V(J)=V(\langle J\rangle)$ for any subset $J$ of $R$.

\item
Let $f\in I(Y)$. Then for all $a\in X$, $a$ is also an element of $Y$ and therefore $f(a)=0$. Thus $f\in I(X)$.

Let $a\in V(J)$. Then for all $f\in I\subseteq J$, $f(a)=0$. Thus $a\in V(I)$.
\end{enumerate}
\end{enumerate}

\chapter{Introduction to Module Theory}

\section{Basic Definitions and Examples}

In these exercises $R$ is a ring with 1 and $M$ is a left $R$-module.

\begin{enumerate}
\item[1.]
These statements are all equivalent to the module being unital. Indeed,
\begin{align*}
1m=m &\iff (0+1)m=m \iff 0m+1m=m \iff 0m+m=m \iff 0m=0 \\
&\iff (-1+1)m=0 \iff (-1)m+m=0 \iff (-1)m=-m.
\end{align*}

\item[2.]
Take $r,s\in R^{\times}$ and some $m\in M$. Then $r(sm) = (rs)m$ because $r$ and $s$ are also in $R$. This shows that the first axiom of a group action is satisfied. Now since $R$ has a 1, take $1\in R^{\times}$. Again, it is easy to see that $1m=m$ and thus the second axiom of a group action is satisfied.

\item[3.]
Suppose there exists some $s\in R$ such that $sr=1$. Then $(sr)m=1m=m$. But we also have $(sr)m=s(rm)=s0=0$. Thus $m=0$, which is contrary to the assumption that $m$ is nonzero. Thus $r$ cannot have an inverse.

Note: for any $r\in R$, $r0 = r(0+0) = r0+r0 \iff r0=0$.

\item[4.]
\begin{enumerate}
\item
Let $N=\{(x_1,x_2,\dots,x_n)\ |\ x_i\in I_i\}$. An ideal of $R$ is also a subgroup of $R$: thus it contains $0$. This means that $(0,\dots,0)\in N$; hence $N$ is not empty. Take any $x,y\in N$ and any $\alpha\in R$. Then $x+\alpha y = (x_1+\alpha y_1,x_2+\alpha y_2,\dots,x_n+\alpha y_n)\in N$ because each $I_i$ is closed under addition and left multiplication by an element of $R$. By the Submodule Criterion, $N$ is a submodule of $M$.
\item
Let $N=\{(x_1,x_2,\dots,x_n)\ |\ x_i\in I_i\ \text{and}\ x_1+x_2+\dots+x_n=0\}$. The proof goes exactly as the last one, except we need to check the sum. We have that $(x_1+\alpha y_1)+(x_2+\alpha y_2)+\dots +(x_n+\alpha y_n) = (x_1+x_2+\dots+x_n)+\alpha(y_1+y_2+\dots+y_n) = 0+\alpha0 = 0$.
\end{enumerate}

\item[5.]
It is clear that $0\in IM$, hence $IM$ is not empty. Without loss of generality, we can take $a_1m_1+a_2m_2+\dots+a_nm_n$ and $b_1m_1+b_2m_2+\dots+b_nm_n$ two elements of $IM$. Take also $\alpha\in R$. Then
\begin{equation*}
\sum_{i=1}^na_im_i+\alpha\sum_{i=1}^nb_im_i = \sum_{i=1}^n(\underbrace{a_i+\alpha b_i}_{\in I})m_i \in IM.
\end{equation*}
Therefore by the Submodule Criterion $IM$ is a submodule of $M$.

\item[6.]
Let $\{M_i\}_{i\in I}$ be a nonempty collection of submodules of an $R$-module. From a result of group theory, we know $M=\bigcap_{i\in I}M_i$ is a subgroup: what's left to check is that $M$ is closed under the action of $R$. Take some $m\in M$. Then $m\in M_i$ for all $i\in I$. Take some $\alpha\in R$. Because each $M_i$ is a module, $\alpha m\in M_i$ $\forall i\in I$. Hence $\alpha m\in M$, proving that $M$ is a submodule of an $R$-module.

\item[7.]
Let $N=\bigcup_{i=1}^\infty N_i$. It is evident that $N$ is nonempty. Pick $x,y\in N$ and $\alpha\in R$. There exists some integers $k$, $l$ such that $x\in N_k$ and $y\in N_l$. Without loss of generality, suppose $k\leq l$. Then $N_k\subseteq N_l$, which means that $x\in N_l$. Because $N_l$ is a module, $x+\alpha y\in N_l\subseteq N$. By the Submodule Criterion, $N$ is a submodule of $M$.

\item[8.]
\begin{enumerate}
\item
It is easy to see that $0 \in \Tor(M)$. Therefore $\Tor(M) \neq \varnothing$. Now let $m,n \in \Tor(M)$ and $\alpha \in R$. There exists nonzero elements $r,s$ of $R$ such that $rm=sn=0$. Thus $rs(m+\alpha n) = (rs)m+(rs\alpha)n = s(rm)+r\alpha(sn) = s0+r\alpha0 = 0$. Since $R$ is an integral domain, the product $rs$ is nonzero. Therefore $m+\alpha n \in \Tor(M)$. By the submodule criterion, $\Tor(M)$ is a submodule of $M$.
\item
Notice that the torsion elements in the $R$-module $R$ are simply the zero divisors of $R$ plus the zero element. Now consider the ring $\mathbb{Z}_6$ as a module over itself. In this module, $2$ and $3$ are torsion elements. However $2+3=5$ is not a torsion element because $5$ is coprime with $6$. Therefore $\Tor(\mathbb{Z}_6)$ is not a subgroup (and thus not a submodule) of $\mathbb{Z}_6$.
\item
Take nonzero elements $a,b$ in $R$ such that $ab=0$. Take some nonzero $m\in M$. If $bm=0$, then $m$ is a torsion element and we are done. Else, $a(bm) = (ab)m = 0m=0$ and $bm$ is a torsion element. In both cases, $\Tor(M)$ is not trivial so the statement is proven.
\end{enumerate}

\item[9.]
Write $I = \{r\in R\ |\ rn = 0\ \forall n\in N\}$ and take $a,b\in I$. Then, for any $n\in N$, $(a+b)n = an+bn=0$, i.e. $a+b \in I$. Now take any $r\in R$. Firstly, $(ra)n = r(an) = r0 = 0$. Secondly, $(ar)n = a(rn)$. Since $rn \in N$ because $N$ is a submodule, and since $a\in I$, we have that $a(rn)=0$. Thus $I$ absorbs multiplication by elements of $R$ on the left and on the right: it is a 2-sided ideal of $R$.

\item[10.]
Write $A=\{m\in M\ |\ am=0\ \forall a\in I\}$. Since it is clear that $0\in A$, we know that $A\neq\varnothing$. Take $m,n\in A$ and $r \in R$. For all $a\in I$, we have that $a(m+bn) = am+a(bn) = (ab)n$. Since $I$ is a right ideal of $R$, $ab\in I$. Thus $(ab)n=0$, meaning that $m+bn\in A$. By the submodule criterion, $A$ is a submodule of $M$.
\end{enumerate}

\section{Quotient Modules and Module Homomorphisms}

\begin{enumerate}
\item[1.]
Let $M$ and $N$ be $R$-modules and let $\varphi:M\to N$ be a $R$-module homomorphism. It is clear that $0\in\ker\varphi$, so it is not empty. Take any $x,y\in\ker\varphi$ and any $r\in R$. Then $\varphi(x+ry) = \varphi(x)+r\varphi(y)=0$ and so $x+ry\in\ker\varphi$. By the submodule criterion, we get that $\ker\varphi$ is a submodule of $M$.
Similarily, we see that $\im\varphi$ is not empty because it contains $0$. Take any $x,y\in \im\varphi$ and any $r\in R$. Then there exists $a,b\in M$ such that $\varphi(a)=x$ and $\varphi(b)=y$. Thus $\varphi(a)+r\varphi(b) = \varphi(a+rb)=x+ry$ and we get by the submodule criterion that $\im\varphi$ is a submodule of $N$.

\item[12.]
The notation in this question seems confusing at first, but realize that $I(R^n)$ and $(IR)^n$ are actually exactly the same thing (and this thing is an $R$-submodule of $R^n$).

We have by Exercise $5$ in Section 1 that $IR$ is a $R$-submodule of $R$. Therefore, by Exercise 11 of this section, we obtain the result immediatly.
\end{enumerate}

\section{Generation of Modules, Direct Sums and Free Modules}

\begin{enumerate}
\item
Notice that a homomorphism $\Phi$ from a free module $F(A)$ to a free module $F(B)$ is necessarily injective. Indeed, if $\sum\alpha_ia_i, \sum\beta_ia_i \in \ker\Phi$, then $\Phi(\sum\alpha_ia_i)=\sum\alpha_i\Phi(a_i)=0$ and $\Phi(\sum\beta_ia_i)=\sum\beta_i\Phi(a_i)=0$. Since $F(B)$ is a free module, $0\in F(B)$ has a unique representation, meaning that $\alpha_i=\beta_i$ for each $i$.

Since $A$ and $B$ are sets of the same cardinality, there exists a bijection $\beta$ between them. Let $i$ and $j$ be inclusion of $A$ in $F(A)$ and of $B$ in $F(B)$ respectively. By Theorem 6, we obtain a unique homomorphism $\Phi:F(A)\to F(B)$ such that $\Phi\circ i = j\circ\beta$. By the previous paragraph, it is a monomorphism. Now take any element $y=\sum\alpha_i(j\circ\beta)(a_i)\in F(B)$. By definition of $\Phi$, we have $\Phi(\sum\alpha_i a_i)=\sum\alpha_i(j\circ\beta)(a_i)=y$ and so $\Phi$ is surjective. Hence it is an isomorphism and $F(A)\cong F(B)$.

\item
Let $I$ be a maximal ideal of $R$ (such a maximal ideal always exists by Zorn's Lemma). Then by Exercise 12 of Section 2, $R^n/IR^n=R^n/I^n\cong (R/I)\times\dots\times(R/I)$ ($n$ times) and similarily $R^m/IR^m\cong (R/I)\times\dots\times(R/I)$ ($m$ times). By maximality of $I$, $R/I=K$ is a field. Thus we get $R^n\cong R^m$ if and only if $K^n\cong K^m$ if and only if $n=m$.

\item
\begin{enumerate}
\item
Consider the $\mathbb{R}[x]$-module $M$ induced from the vector space $\mathbb{R}^2$ over the field $\mathbb{R}$ using the linear transformation $T$ which sends a vector to its counter-clockwise rotation by $\pi/2$ radians. Take $(1,0)\in \mathbb{R}$: this element is a generator for $M$. Indeed, take any $(a,b)\in \mathbb{R}^2$. Then $(a+bx)\cdot(1,0) = a\cdot(1,0)+b\cdot T(1,0) = a(1,0)+b(0,1)=(a,b)$, hence $\mathbb{R}[x]\cdot(1,0)=M$. Therefore $M$ is a cyclic module.
\item
Consider a similar $M$ again but this time induced using the linear transformation $T'$ which is a projection on the $y$-axis. The element $(1,1)\in\mathbb{R}^2$ generates $M$: take any $(a,b)\in\mathbb{R}^2$. Then $(a+(b-a)x)\cdot(1,1)=a(1,1)+(b-a)T'(1,1)=a(1,1)+(b-a)(0,1)=(a,a)+(0,b-a)=(a,b)$. Thus $M$ is a cyclic module.
\end{enumerate}

\item[9.]
Suppose that $M\neq 0$ and $M$ is a cyclic module with any nonzero element as generator. Take $N\neq 0$ a submodule of $M$ and pick some $n\in N$. Then $Rn\subseteq N$ as $N$ is closed under the action of $R$. Moreover $Rn=M$ by our supposition. Hence $M=N$. Because $N$ was aribtrary, we conclude that $M$ is irreducible. On the other hand, suppose that $M$ is irreducible (and so $M\neq 0$). Take any nonzero $m\in M$. Then $Rm$ is a submodule of $M$, hence by irreducibility $Rm=M$ and $m$ is a generator of $M$.

Because $\mathbb{Z}$-modules are the same thing as abelian groups and $\mathbb{Z}$-submodules are the same thing as subgroups of abelian groups, the irreducible $\mathbb{Z}$-modules are exactly the simple abelian groups. By basic group theory, these are exactly the abelian groups having order a prime number.

\item[11.] \textbf{Schur's Lemma}. Take $M_1$ and $M_2$ irreducible $R$-modules and $\varphi\in\Hom_R(M_1,M_2)$ with $\varphi$ nonzero. Since $\ker\varphi$ is a submodule of $M_1$ and $\ker\varphi\neq M_1$, we must have $\ker\varphi=0$. Similarily, since $\im\varphi$ is a submodule of $M_2$ and $\im\varphi\neq 0$, we must have $\im\varphi=M_2$. Thus $M_1\cong M_2$. Now consider some $\alpha\in\End_R(M)$ for $M$ an irreducible $R$-module. By the previous result, $\alpha$ must be an automorphism or the zero homomorphism; in the first case it always has an inverse. Therefore $\End_R(M)$ is a divison ring.
\end{enumerate}

\end{document}

