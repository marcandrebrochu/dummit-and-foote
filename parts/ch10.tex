\chapter{Introduction to Module Theory}

\section{Basic Definitions and Examples}

In these exercises $R$ is a ring with 1 and $M$ is a left $R$-module.

\answer{1}{
These statements are all equivalent to the module being unital. Indeed,
\begin{align*}
1m=m &\iff (0+1)m=m \iff 0m+1m=m \iff 0m+m=m \iff 0m=0 \\
&\iff (-1+1)m=0 \iff (-1)m+m=0 \iff (-1)m=-m.
\end{align*}
}

\answer{2}{
Take $r,s\in R^{\times}$ and some $m\in M$. Then $r(sm) = (rs)m$ because $r$ and $s$ are also in $R$. This shows that the first axiom of a group action is satisfied. Now since $R$ has a 1, take $1\in R^{\times}$. Again, it is easy to see that $1m=m$ and thus the second axiom of a group action is satisfied.
}

\answer{3}{
Suppose there exists some $s\in R$ such that $sr=1$. Then $(sr)m=1m=m$. But we also have $(sr)m=s(rm)=s0=0$. Thus $m=0$, which is contrary to the assumption that $m$ is nonzero. Thus $r$ cannot have an inverse.

Note: for any $r\in R$, $r0 = r(0+0) = r0+r0 \iff r0=0$.
}

\answer{4}{
\begin{enumerate}[label=(\alph*)]
\item
Let $N=\{(x_1,x_2,\dots,x_n)\ |\ x_i\in I_i\}$. An ideal of $R$ is also a subgroup of $R$: thus it contains $0$. This means that $(0,\dots,0)\in N$; hence $N$ is not empty. Take any $x,y\in N$ and any $\alpha\in R$. Then $x+\alpha y = (x_1+\alpha y_1,x_2+\alpha y_2,\dots,x_n+\alpha y_n)\in N$ because each $I_i$ is closed under addition and left multiplication by an element of $R$. By the Submodule Criterion, $N$ is a submodule of $M$.
\item
Let $N=\{(x_1,x_2,\dots,x_n)\ |\ x_i\in I_i\ \text{and}\ x_1+x_2+\dots+x_n=0\}$. The proof goes exactly as the last one, except we need to check the sum. We have that $(x_1+\alpha y_1)+(x_2+\alpha y_2)+\dots +(x_n+\alpha y_n) = (x_1+x_2+\dots+x_n)+\alpha(y_1+y_2+\dots+y_n) = 0+\alpha0 = 0$.
\end{enumerate}
}

\answer{5}{
It is clear that $0\in IM$, hence $IM$ is not empty. Without loss of generality, we can take $a_1m_1+a_2m_2+\dots+a_nm_n$ and $b_1m_1+b_2m_2+\dots+b_nm_n$ two elements of $IM$. Take also $\alpha\in R$. Then
\begin{equation*}
\sum_{i=1}^na_im_i+\alpha\sum_{i=1}^nb_im_i = \sum_{i=1}^n(\underbrace{a_i+\alpha b_i}_{\in I})m_i \in IM.
\end{equation*}
Therefore by the Submodule Criterion $IM$ is a submodule of $M$.
}

\answer{6}{
Let $\{M_i\}_{i\in I}$ be a nonempty collection of submodules of an $R$-module. From a result of group theory, we know $M=\bigcap_{i\in I}M_i$ is a subgroup: what's left to check is that $M$ is closed under the action of $R$. Take some $m\in M$. Then $m\in M_i$ for all $i\in I$. Take some $\alpha\in R$. Because each $M_i$ is a module, $\alpha m\in M_i$ $\forall i\in I$. Hence $\alpha m\in M$, proving that $M$ is a submodule of an $R$-module.
}

\answer{7}{
Let $N=\bigcup_{i=1}^\infty N_i$. It is evident that $N$ is nonempty. Pick $x,y\in N$ and $\alpha\in R$. There exists some integers $k$, $l$ such that $x\in N_k$ and $y\in N_l$. Without loss of generality, suppose $k\leq l$. Then $N_k\subseteq N_l$, which means that $x\in N_l$. Because $N_l$ is a module, $x+\alpha y\in N_l\subseteq N$. By the Submodule Criterion, $N$ is a submodule of $M$.
}

\answer{8}{
\begin{enumerate}[label=(\alph*)]
\item
It is easy to see that $0 \in \Tor(M)$. Therefore $\Tor(M) \neq \varnothing$. Now let $m,n \in \Tor(M)$ and $\alpha \in R$. This means there exists nonzero elements $r,s$ of $R$ such that $rm=sn=0$. Since $R$ is an integral domain, the product $rs$ is nonzero. We compute $rs(m+\alpha n) = (rs)m+(rs\alpha)n = s(rm)+r\alpha(sn) = s0+r\alpha0 = 0$. Therefore $m+\alpha n \in \Tor(M)$. By the submodule criterion, $\Tor(M)$ is a submodule of $M$.
\item
Notice that the torsion elements in the $R$-module $R$ are simply the zero divisors of $R$ plus the zero element. For instance, consider the ring $\mathbb{Z}_6$ as a module over itself. Its zero divisors (torsion elements) are $\{0,2,3,4\}$, and so it is clear that $\Tor(\mathbb{Z}_6)$ fails to be a group: $2+3=5$ is not a torsion element because $5$ is coprime with $6$ and hence is not a zero divisor. Therefore $\Tor(\mathbb{Z}_6)$ is not a submodule of $\mathbb{Z}_6$.
\item
Take nonzero elements $a,b$ in $R$ such that $ab=0$. Take some nonzero $m\in M$. If $bm=0$, then $m$ is a torsion element and we are done. Else, $a(bm) = (ab)m = 0m=0$ and $bm$ is a torsion element. In both cases, $\Tor(M)$ is not trivial so the statement is proven.
\end{enumerate}
}

\answer{9}{
Write $I = \{r\in R\ |\ rn = 0\ \forall n\in N\}$ and take $a,b\in I$. Then, for any $n\in N$, $(a+b)n = an+bn=0$, i.e. $a+b \in I$. Now take any $r\in R$. Firstly, $(ra)n = r(an) = r0 = 0$. Secondly, $(ar)n = a(rn)$. Since $rn \in N$ because $N$ is a submodule, and since $a\in I$, we have that $a(rn)=0$. Thus $I$ absorbs multiplication by elements of $R$ on the left and on the right: it is a 2-sided ideal of $R$.
}

\answer{10}{
Write $A=\{m\in M\ |\ am=0\ \forall a\in I\}$. Since it is clear that $0\in A$, we know that $A\neq\varnothing$. Take $m,n\in A$ and $r \in R$. For all $a\in I$, we have that $a(m+bn) = am+a(bn) = (ab)n$. Since $I$ is a right ideal of $R$, $ab\in I$. Thus $(ab)n=0$, meaning that $m+bn\in A$. By the submodule criterion, $A$ is a submodule of $M$.
}

\section{Quotient Modules and Module Homomorphisms}

\answer{1}{
Let $M$ and $N$ be $R$-modules and let $\varphi:M\to N$ be a $R$-module homomorphism. It is clear that $0\in\ker\varphi$, so it is not empty. Take any $x,y\in\ker\varphi$ and any $r\in R$. Then $\varphi(x+ry) = \varphi(x)+r\varphi(y)=0$ and so $x+ry\in\ker\varphi$. By the submodule criterion, we get that $\ker\varphi$ is a submodule of $M$. Similarily, we see that $\im\varphi$ is not empty because it contains $0$. Take any $x,y\in \im\varphi$ and any $r\in R$. Then there exists $a,b\in M$ such that $\varphi(a)=x$ and $\varphi(b)=y$. Thus $\varphi(a)+r\varphi(b) = \varphi(a+rb)=x+ry$ and we get by the submodule criterion that $\im\varphi$ is a submodule of $N$.
}

\answer{8}{
Pick some element $x\in\Tor(M)$. Then there exists some nonzero $r\in R$ such that $rx=0$. Hence $\varphi(rx)=r\varphi(x)=0$ because $\varphi$ is a homomorphism and thus preserves the action of $r$ and maps $0_M$ to $0_N$. This means that $\varphi(x)\in\Tor(N)$ (the element of $R$ that annihilates $x$ in $M$ also annihilates $\varphi(x)$ in $N$). Therefore $\varphi(\Tor(M))\subseteq\Tor(N)$.
}

\answer{9}{
Following the hint in the text, we wish to show that each $\varphi\in\Hom_R(R,M)$ is completely determined by its value at $1\in R$. This is easy to see: $\varphi(r)=\varphi(r\cdot 1)=r\varphi(1)$ for all $r\in R$ (the action of $R$ on itself as a module is just multiplication). This means that two different $\varphi$, $\nu \in \Hom_R(R,M)$ will have $\varphi(1)\neq\nu(1)$. Hence the map $f:\Hom_R(R,M)\to M$ given by $f(\varphi)=\varphi(1)$ is injective.

To show that the map $f$ is surjective, take any $m\in M$ and consider the mapping $\varphi_m:R\to M$ given by $\varphi_m(r)=rm$. It is easy to see that this mapping is a homomorphism between $R$-modules. Let us check just to be sure: if we have $x,y,\alpha\in R$, then $\varphi_m(\alpha x+y) = (\alpha x+y)m = (\alpha x)m+ym=\alpha(xm)+ym=\alpha\varphi_m(x)+\varphi_m(y)$, which verifies our claim. Moreover, $f(\varphi_m)=\varphi_m(1)=1m=m$. This shows that $\text{im}\,f=M$.

Now we need to show that $f$ is a ($R$-module) homomorphism. Because it is a bijection by the preceding two paragraphs, this will prove that $\Hom_R(R,M)\cong M$. Take $\alpha\in R$ and $\varphi, \nu\in \Hom_R(R,M)$. Then $f(\alpha\varphi+\nu) = (\alpha\varphi+\nu)(1)=(\alpha\varphi)(1)+\nu(1)=\alpha(\varphi(1))+\nu(1)=\alpha f(\varphi)+f(\nu)$. Hence $f$ is an ($R$-module) isomorphism. This finishes the answer to the question.
}

\answer{12}{
The notation in this question seems confusing at first, but realize that $I(R^n)$ and $(IR)^n$ are actually exactly the same thing (and this thing is an $R$-submodule of $R^n$).

We have by Exercise $5$ in Section 1 that $IR$ is a $R$-submodule of $R$. Therefore, by Exercise 11 of this section, we obtain the result immediatly.
}

\section{Generation of Modules, Direct Sums and Free Modules}

\answer{2}{
Let $I$ be a maximal ideal of $R$ (such a maximal ideal always exists by Zorn's Lemma). Then by Exercise 12 of Section 2, $R^n/IR^n=R^n/I^n\cong (R/I)\times\dots\times(R/I)$ ($n$ times) and similarily $R^m/IR^m\cong (R/I)\times\dots\times(R/I)$ ($m$ times). By maximality of $I$, $R/I=K$ is a field. Thus we get $R^n\cong R^m$ if and only if $K^n\cong K^m$ if and only if $n=m$.
}

\answer{3}{
\begin{enumerate}[label=(\alph*)]
\item
Consider the $\mathbb{R}[x]$-module $M$ induced from the vector space $\mathbb{R}^2$ over the field $\mathbb{R}$ using the linear transformation $T$ which sends a vector to its counter-clockwise rotation by $\pi/2$ radians. Take $(1,0)\in \mathbb{R}$: this element is a generator for $M$. Indeed, take any $(a,b)\in \mathbb{R}^2$. Then $(a+bx)\cdot(1,0) = a\cdot(1,0)+b\cdot T(1,0) = a(1,0)+b(0,1)=(a,b)$, hence $\mathbb{R}[x]\cdot(1,0)=M$. Therefore $M$ is a cyclic module.
\item
Consider a similar $M$ again but this time induced using the linear transformation $T'$ which is a projection on the $y$-axis. The element $(1,1)\in\mathbb{R}^2$ generates $M$: take any $(a,b)\in\mathbb{R}^2$. Then $(a+(b-a)x)\cdot(1,1)=a(1,1)+(b-a)T'(1,1)=a(1,1)+(b-a)(0,1)=(a,a)+(0,b-a)=(a,b)$. Thus $M$ is a cyclic module.
\end{enumerate}
}

\answer{7}{
Take $A=\{\overline{a_1},\overline{a_2},\dots,\overline{a_n}\}$ to be a generating set for $M/N$ and $B=\{b_1,b_2,\dots,b_m\}$ to be a generating set for $N$. Pick any element $m\in M$. We will show that this element can be written using only the (finite number of) generators in $A\cup B$. This will show that $M$ is a finitely generated module.

As usual, $\overline{m}$ denotes the projection of $m$ inside $M/N$. We have that $\overline{m}=r_1\overline{a_1}+r_2\overline{a_2}+\dots+r_n\overline{a_n}=\overline{r_1a_1+r_2a_2+\dots+r_na_n}$. This holds if and only if $m-(r_1a_1+r_2a_2+\dots+r_na_n)\in N$, and so $m-(r_1a_1+r_2a_2+\dots+r_na_n)=s_1b_1+s_2b_2+\dots+s_mb_m$. Hence $m=r_1a_1+r_2a_2+\dots+r_na_n+s_1b_1+s_2b_2+\dots+s_mb_m$. Because $m$ was arbitrary, this proves the claim that $M$ is a finitely generated module and $M=R(A\cup B)$.
}

\answer{9}{
Suppose that $M\neq 0$ and $M$ is a cyclic module with any nonzero element as generator. Take $N\neq 0$ a submodule of $M$ and pick some $n\in N$. Then $Rn\subseteq N$ as $N$ is closed under the action of $R$. Moreover $Rn=M$ by our supposition. Hence $M=N$. Because $N$ was aribtrary, we conclude that $M$ is irreducible. On the other hand, suppose that $M$ is irreducible (and so $M\neq 0$). Take any nonzero $m\in M$. Then $Rm$ is a submodule of $M$, hence by irreducibility $Rm=M$ and $m$ is a generator of $M$.

Because $\mathbb{Z}$-modules are the same thing as abelian groups and $\mathbb{Z}$-submodules are the same thing as subgroups of abelian groups, the irreducible $\mathbb{Z}$-modules are exactly the simple abelian groups. By basic group theory, these are exactly the abelian groups having order a prime number.
}

\answer{11}{
\textbf{Schur's Lemma}. Take $M_1$ and $M_2$ irreducible $R$-modules and $\varphi\in\Hom_R(M_1,M_2)$ with $\varphi$ nonzero. Since $\ker\varphi$ is a submodule of $M_1$ and $\ker\varphi\neq M_1$, we must have $\ker\varphi=0$. Similarily, since $\im\varphi$ is a submodule of $M_2$ and $\im\varphi\neq 0$, we must have $\im\varphi=M_2$. Thus $M_1\cong M_2$. Now consider some $\alpha\in\End_R(M)$ for $M$ an irreducible $R$-module. By the previous result, $\alpha$ must be an automorphism or the zero homomorphism; in the first case it always has an inverse. Therefore $\End_R(M)$ is a divison ring.
}
