\chapter{Polynomial Rings}

\section{Definitions and Basic Properties}

\answer{12}{
First, notice that in $R$, the ideal $(x,z)$ is equal to the ideal $(x,z,xy-z^2)$. This is pretty easy to see but we will prove it explicitely. Clearly, $(x,z)\subseteq (x,z,xy-z^2)$. Now take some $a\in(x,z,xy-z^2)$. Then $a=r_1x+r_2z+r_3xy-r_3z^2$ for some $r_1,r_2,r_3\in R$. Hence $a=(r_1+r_3y)x+(r_2+r_3z)z\in(x,z)$. Therefore $(x,z,xy-z^2)=(x,z)$.

Let $I=(xy-z^2)$ and $J=(x,z,xy-z^2)=(x,z)$. We have that $I\subset J$. By the Third Isomorphism Theorem for Rings (p.246 of D\&F), we get that $J/I=(\overline{x},\overline{z})=\overline{P}$ is an ideal of $R/I=\overline{R}$ (we already knew that) and crucially,
\begin{equation*}
\overline{R}/\overline{P}\cong R/J=\mathbb{Q}[x,y,z]/(x,z)\cong \mathbb{Q}[y].
\end{equation*}
Because $\mathbb{Q}[y]$ is an integral domain, we conclude that the ideal $\overline{P}$ of $\overline{R}$ must be a prime ideal.

Since $\overline{P}^2=\{\overline{\alpha}\cdot\overline{\gamma}\ |\ \overline{\alpha},\overline{\gamma}\in\overline{P}\}$, it is clear that $\overline{xy}=\overline{z}\cdot \overline{z}\in \overline{P}^2$. Now suppose that there exists some integer $k\geq 0$ such that $\overline{y}^k\in \overline{P}^2$. This means that $\overline{y}^k=\overline{\alpha}\cdot\overline{\gamma}$ for some $\overline{\alpha},\overline{\gamma}\in\overline{P}$. Let $\pi:\overline{R}\to\overline{R}/\overline{P}$ be the natural projection homomorphism. Then $\pi(\overline{y}^k)=\pi(\overline{y})^k=0$. However $\pi(\overline{y})$ is nonzero. This is in contradiction with the fact that $\mathbb{Q}[y]$, to which $\overline{R}/\overline{P}$ is isomorphic, is an integral domain. Therefore no power of $\overline{y}$ lies in $\overline{P}^2$.
}

\section{Poynomial Rings Over Fields I}

\section{Polynomial Rings that are Unique Factorization Domains}

\answer{2}{
\begin{enumerate}[label=(\alph*)]
\item We apply Eisenstein's Criterion with prime $p=2$. We can apply the criterion because $2$ divides both $-4$ and $6$, but happily $p^2=4$ does not divide $6$ and thus $x^4-4x^3+6$ is irreducible in $\mathbb{Z}[x]$.

\item This is another direct application of Eisenstein's Criterion, using prime $p=3$. Indeed, $3$ divides all of $30$, $-15$, $6$ and $-120$ while $p^2=9$ fails to divide $120$. Hence $x^6+30x^5-15x^3+6x-120$ is irreducible in $\mathbb{Z}[x]$.

\item Let $f(x)=x^4+4x^3+6x^2+2x+1$. It seems we cannot apply Eisenstein's Criterion directly, but following the hint in the question we let $g(x)=f(x-1)$. After computation, we get that $g(x)=x^4-2x+2$. We apply Eisenstein's Criterion with prime $p=2$ on $g(x)$ to obtain that $g(x)$ is irreducible in $\mathbb{Z}[x]$. This means that $f(x)$ is also irreducible: if it were not, then the reduction $f(x)=a(x)b(x)$ would make $g(x)$ reducible because we would have $g(x)=a(x-1)b(x-1)$.

\item We compute
\begin{equation*}
f(x)=\frac{(x+2)^p-2^p}{x} = \sum_{k=1}^p {p\choose k}x^{k-1}2^{p-k} = p\,2^{p-1}+x^{p-1}+\sum_{k=2}^{p-1}{p\choose k}x^{k-1}2^{p-k}
\end{equation*}
and we can now see that the prime integer $p$ divides every coefficient of $f(x)$ except the leading one (which is 1, i.e. this polynomial is monic). Moreover, because $p$ is odd, we see that $p^2$ cannot divide $p\,2^{p-1}$. Therefore, by the Eisenstein Criterion, $f(x)$ is irreducible in $\mathbb{Z}[x]$.
\end{enumerate}
}

\section{Polynomial Rings over Fields II}

For the remaining exercises let $F$ be a field, let $F^n$ be the set of all $n$-tuples of elements of $F$ (called \textit{affine $n$-space over F}) and let $R$ be the polynomial ring $F[x_1,x_2,\dots,x_n]$. The elements of $R$ form a ring of $F$-valued functions on $F^n$, where the value of the polynomial $p(x_1,\dots,x_n)$ on the $n$-tuple $(a_1,\dots,a_n)$ is obtained by substituting $a_i$ for $x_i$ for all $i$.

\answer{12}{
\begin{enumerate}
\item
Let $X$ be any given subset of $F^n$. We always have $0_R\in I(X)$ and thus $I(X)$ is never empty. Take any $f,g\in I(X)$. Then for all $a\in X$, $(f+g)(x) = f(x)+g(x)=0$. Thus $I(X)$ is closed under addition. Take some $h\in R$. For all $a\in X$, $(h\cdot f)(x)=h(x)f(x)=0$ which means that $I(X)$ absorbs left multiplication. Because $R$ is commutative, we get that $I(X)$ is an ideal in this ring.

Let $J\subseteq R$ be arbitrarily given. If $a\in V(\langle J\rangle)$, then for all $f\in J\subseteq \langle J\rangle$, we have that $f(a)=0$. Thus $V(\langle J\rangle)\subseteq V(J)$. Now let $a\in V(J)$. Take any $f\in \langle J\rangle$. Then $f$ is a finite combination of $R$-multiples of elements of $J$, i.e. $f=f_1j_1+\dots+f_nj_n$ with $f_i\in R$, $j_i\in J$ and $n\in \mathbb{N}$. So $f(a) = f_1(a)j_1(a)+\dots+f_n(a)j_n(a)$. Since for all $j\in J$, $j(a)=0$, we get that $f(a)=0$ and thus $a\in V(\langle J\rangle)$. Therefore $V(J)=V(\langle J\rangle)$ for any subset $J$ of $R$.

\item
Let $f\in I(Y)$. Then for all $a\in X$, $a$ is also an element of $Y$ and therefore $f(a)=0$. Thus $f\in I(X)$.

Let $a\in V(J)$. Then for all $f\in I\subseteq J$, $f(a)=0$. Thus $a\in V(I)$.
\end{enumerate}
}
